% Options for packages loaded elsewhere
\PassOptionsToPackage{unicode}{hyperref}
\PassOptionsToPackage{hyphens}{url}
%
\documentclass[
]{article}
\usepackage{amsmath,amssymb}
\usepackage{iftex}
\ifPDFTeX
  \usepackage[T1]{fontenc}
  \usepackage[utf8]{inputenc}
  \usepackage{textcomp} % provide euro and other symbols
\else % if luatex or xetex
  \usepackage{unicode-math} % this also loads fontspec
  \defaultfontfeatures{Scale=MatchLowercase}
  \defaultfontfeatures[\rmfamily]{Ligatures=TeX,Scale=1}
\fi
\usepackage{lmodern}
\ifPDFTeX\else
  % xetex/luatex font selection
\fi
% Use upquote if available, for straight quotes in verbatim environments
\IfFileExists{upquote.sty}{\usepackage{upquote}}{}
\IfFileExists{microtype.sty}{% use microtype if available
  \usepackage[]{microtype}
  \UseMicrotypeSet[protrusion]{basicmath} % disable protrusion for tt fonts
}{}
\makeatletter
\@ifundefined{KOMAClassName}{% if non-KOMA class
  \IfFileExists{parskip.sty}{%
    \usepackage{parskip}
  }{% else
    \setlength{\parindent}{0pt}
    \setlength{\parskip}{6pt plus 2pt minus 1pt}}
}{% if KOMA class
  \KOMAoptions{parskip=half}}
\makeatother
\usepackage{xcolor}
\usepackage[margin=1in]{geometry}
\usepackage{color}
\usepackage{fancyvrb}
\newcommand{\VerbBar}{|}
\newcommand{\VERB}{\Verb[commandchars=\\\{\}]}
\DefineVerbatimEnvironment{Highlighting}{Verbatim}{commandchars=\\\{\}}
% Add ',fontsize=\small' for more characters per line
\usepackage{framed}
\definecolor{shadecolor}{RGB}{248,248,248}
\newenvironment{Shaded}{\begin{snugshade}}{\end{snugshade}}
\newcommand{\AlertTok}[1]{\textcolor[rgb]{0.94,0.16,0.16}{#1}}
\newcommand{\AnnotationTok}[1]{\textcolor[rgb]{0.56,0.35,0.01}{\textbf{\textit{#1}}}}
\newcommand{\AttributeTok}[1]{\textcolor[rgb]{0.13,0.29,0.53}{#1}}
\newcommand{\BaseNTok}[1]{\textcolor[rgb]{0.00,0.00,0.81}{#1}}
\newcommand{\BuiltInTok}[1]{#1}
\newcommand{\CharTok}[1]{\textcolor[rgb]{0.31,0.60,0.02}{#1}}
\newcommand{\CommentTok}[1]{\textcolor[rgb]{0.56,0.35,0.01}{\textit{#1}}}
\newcommand{\CommentVarTok}[1]{\textcolor[rgb]{0.56,0.35,0.01}{\textbf{\textit{#1}}}}
\newcommand{\ConstantTok}[1]{\textcolor[rgb]{0.56,0.35,0.01}{#1}}
\newcommand{\ControlFlowTok}[1]{\textcolor[rgb]{0.13,0.29,0.53}{\textbf{#1}}}
\newcommand{\DataTypeTok}[1]{\textcolor[rgb]{0.13,0.29,0.53}{#1}}
\newcommand{\DecValTok}[1]{\textcolor[rgb]{0.00,0.00,0.81}{#1}}
\newcommand{\DocumentationTok}[1]{\textcolor[rgb]{0.56,0.35,0.01}{\textbf{\textit{#1}}}}
\newcommand{\ErrorTok}[1]{\textcolor[rgb]{0.64,0.00,0.00}{\textbf{#1}}}
\newcommand{\ExtensionTok}[1]{#1}
\newcommand{\FloatTok}[1]{\textcolor[rgb]{0.00,0.00,0.81}{#1}}
\newcommand{\FunctionTok}[1]{\textcolor[rgb]{0.13,0.29,0.53}{\textbf{#1}}}
\newcommand{\ImportTok}[1]{#1}
\newcommand{\InformationTok}[1]{\textcolor[rgb]{0.56,0.35,0.01}{\textbf{\textit{#1}}}}
\newcommand{\KeywordTok}[1]{\textcolor[rgb]{0.13,0.29,0.53}{\textbf{#1}}}
\newcommand{\NormalTok}[1]{#1}
\newcommand{\OperatorTok}[1]{\textcolor[rgb]{0.81,0.36,0.00}{\textbf{#1}}}
\newcommand{\OtherTok}[1]{\textcolor[rgb]{0.56,0.35,0.01}{#1}}
\newcommand{\PreprocessorTok}[1]{\textcolor[rgb]{0.56,0.35,0.01}{\textit{#1}}}
\newcommand{\RegionMarkerTok}[1]{#1}
\newcommand{\SpecialCharTok}[1]{\textcolor[rgb]{0.81,0.36,0.00}{\textbf{#1}}}
\newcommand{\SpecialStringTok}[1]{\textcolor[rgb]{0.31,0.60,0.02}{#1}}
\newcommand{\StringTok}[1]{\textcolor[rgb]{0.31,0.60,0.02}{#1}}
\newcommand{\VariableTok}[1]{\textcolor[rgb]{0.00,0.00,0.00}{#1}}
\newcommand{\VerbatimStringTok}[1]{\textcolor[rgb]{0.31,0.60,0.02}{#1}}
\newcommand{\WarningTok}[1]{\textcolor[rgb]{0.56,0.35,0.01}{\textbf{\textit{#1}}}}
\usepackage{graphicx}
\makeatletter
\def\maxwidth{\ifdim\Gin@nat@width>\linewidth\linewidth\else\Gin@nat@width\fi}
\def\maxheight{\ifdim\Gin@nat@height>\textheight\textheight\else\Gin@nat@height\fi}
\makeatother
% Scale images if necessary, so that they will not overflow the page
% margins by default, and it is still possible to overwrite the defaults
% using explicit options in \includegraphics[width, height, ...]{}
\setkeys{Gin}{width=\maxwidth,height=\maxheight,keepaspectratio}
% Set default figure placement to htbp
\makeatletter
\def\fps@figure{htbp}
\makeatother
\setlength{\emergencystretch}{3em} % prevent overfull lines
\providecommand{\tightlist}{%
  \setlength{\itemsep}{0pt}\setlength{\parskip}{0pt}}
\setcounter{secnumdepth}{-\maxdimen} % remove section numbering
\ifLuaTeX
  \usepackage{selnolig}  % disable illegal ligatures
\fi
\usepackage{bookmark}
\IfFileExists{xurl.sty}{\usepackage{xurl}}{} % add URL line breaks if available
\urlstyle{same}
\hypersetup{
  pdftitle={Statistical Inference assignment},
  pdfauthor={success},
  hidelinks,
  pdfcreator={LaTeX via pandoc}}

\title{Statistical Inference assignment}
\author{success}
\date{2024-09-01}

\begin{document}
\maketitle

\begin{Shaded}
\begin{Highlighting}[]
\FunctionTok{library}\NormalTok{(ggplot2)}
\FunctionTok{library}\NormalTok{(dplyr)}
\end{Highlighting}
\end{Shaded}

\subsubsection{Overview}\label{overview}

In this article we are going to the exploring the \textbf{ToothGrowth}
data set in R. The goal is to explore the variables in the data set and
see if we can draw any reasonable inference about ToothGrowth in guinea
pigs.

\subsubsection{The data}\label{the-data}

\begin{Shaded}
\begin{Highlighting}[]
\FunctionTok{data}\NormalTok{(}\StringTok{"ToothGrowth"}\NormalTok{)}
\NormalTok{data }\OtherTok{=}\NormalTok{ ToothGrowth}
\FunctionTok{str}\NormalTok{(data)}
\end{Highlighting}
\end{Shaded}

\begin{verbatim}
## 'data.frame':    60 obs. of  3 variables:
##  $ len : num  4.2 11.5 7.3 5.8 6.4 10 11.2 11.2 5.2 7 ...
##  $ supp: Factor w/ 2 levels "OJ","VC": 2 2 2 2 2 2 2 2 2 2 ...
##  $ dose: num  0.5 0.5 0.5 0.5 0.5 0.5 0.5 0.5 0.5 0.5 ...
\end{verbatim}

The dataset consist of 60 observations in 3 columns. The supp variable
represent vitamin supplements given to guinea pigs. The length len
represents tooth growth and the dose represents dosage

\subsubsection{Exploratory Data
Analysis}\label{exploratory-data-analysis}

We expect len to increase with dose. Figure below is a box plot of len
for each of dose.

\begin{Shaded}
\begin{Highlighting}[]
\NormalTok{data}\SpecialCharTok{$}\NormalTok{dose }\OtherTok{=} \FunctionTok{as.factor}\NormalTok{(data}\SpecialCharTok{$}\NormalTok{dose)}

\FunctionTok{ggplot}\NormalTok{(data, }\FunctionTok{aes}\NormalTok{(}\FunctionTok{factor}\NormalTok{(dose), len)) }\SpecialCharTok{+} 
  \FunctionTok{geom\_boxplot}\NormalTok{()}
\end{Highlighting}
\end{Shaded}

\includegraphics{analysis2_files/figure-latex/unnamed-chunk-3-1.pdf}

Now if we plot, make of similar box plot of len against supp, we see
that average tooth growth when supplement 1 is given is greater than the
average tooth growth when supplement 2 is given.

\begin{Shaded}
\begin{Highlighting}[]
\FunctionTok{ggplot}\NormalTok{(data, }\FunctionTok{aes}\NormalTok{(}\FunctionTok{factor}\NormalTok{(supp), len)) }\SpecialCharTok{+} 
  \FunctionTok{geom\_boxplot}\NormalTok{()}
\end{Highlighting}
\end{Shaded}

\includegraphics{analysis2_files/figure-latex/unnamed-chunk-4-1.pdf}

\subsubsection{Stastical Inference}\label{stastical-inference}

Of course, this could be due to random chance so I am going to formally
test the hypothesis that their averages are the same using a Welch Two
Sample t-test at a type one error rate of 0.05

\begin{Shaded}
\begin{Highlighting}[]
\NormalTok{ojdata }\OtherTok{\textless{}{-}}\NormalTok{ dplyr}\SpecialCharTok{::}\FunctionTok{filter}\NormalTok{(data, supp}\SpecialCharTok{==}\StringTok{\textquotesingle{}OJ\textquotesingle{}}\NormalTok{)}
\NormalTok{vcdata }\OtherTok{\textless{}{-}}\NormalTok{ dplyr}\SpecialCharTok{::}\FunctionTok{filter}\NormalTok{(data, supp}\SpecialCharTok{==}\StringTok{\textquotesingle{}VC\textquotesingle{}}\NormalTok{)}
\NormalTok{test1 }\OtherTok{\textless{}{-}} \FunctionTok{t.test}\NormalTok{(ojdata}\SpecialCharTok{$}\NormalTok{len, vcdata}\SpecialCharTok{$}\NormalTok{len, }\AttributeTok{paired =} \ConstantTok{TRUE}\NormalTok{)}
\NormalTok{test1}
\end{Highlighting}
\end{Shaded}

\begin{verbatim}
## 
##  Paired t-test
## 
## data:  ojdata$len and vcdata$len
## t = 3.3026, df = 29, p-value = 0.00255
## alternative hypothesis: true mean difference is not equal to 0
## 95 percent confidence interval:
##  1.408659 5.991341
## sample estimates:
## mean difference 
##             3.7
\end{verbatim}

In this case, I made the \textbf{assumption that the observations in the
different groups are paired}. The p-value of the Welch test is
0.0025498. Hence we reject the null hypothesis at a 95\% significance.
\textbf{The average of the group that took the `OJ' supplement is
greater than the mean of the group that took the `VC' supplement}.

Similarly, we can also run a Welch t-test for the significance of the
observed difference in mean. Here we also make the \textbf{assumption
that observations are paired}.

\begin{Shaded}
\begin{Highlighting}[]
\NormalTok{group1 }\OtherTok{\textless{}{-}}\NormalTok{ dplyr}\SpecialCharTok{::}\FunctionTok{filter}\NormalTok{(data, dose}\SpecialCharTok{==}\FloatTok{0.5}\NormalTok{)}
\NormalTok{group2 }\OtherTok{\textless{}{-}}\NormalTok{ dplyr}\SpecialCharTok{::}\FunctionTok{filter}\NormalTok{(data, dose}\SpecialCharTok{==}\DecValTok{1}\NormalTok{)}
\NormalTok{group3 }\OtherTok{\textless{}{-}}\NormalTok{ dplyr}\SpecialCharTok{::}\FunctionTok{filter}\NormalTok{(data, dose}\SpecialCharTok{==}\DecValTok{2}\NormalTok{)}

\NormalTok{test2 }\OtherTok{\textless{}{-}} \FunctionTok{t.test}\NormalTok{(group1}\SpecialCharTok{$}\NormalTok{len, group2}\SpecialCharTok{$}\NormalTok{len, }\AttributeTok{paired=}\ConstantTok{TRUE}\NormalTok{)}
\NormalTok{test3 }\OtherTok{\textless{}{-}} \FunctionTok{t.test}\NormalTok{(group2}\SpecialCharTok{$}\NormalTok{len, group3}\SpecialCharTok{$}\NormalTok{len, }\AttributeTok{paired=}\ConstantTok{TRUE}\NormalTok{)}

\NormalTok{test2}\SpecialCharTok{$}\NormalTok{p.value }\SpecialCharTok{\textless{}} \FloatTok{0.05}
\NormalTok{test3}\SpecialCharTok{$}\NormalTok{p.value }\SpecialCharTok{\textless{}} \FloatTok{0.05}
\end{Highlighting}
\end{Shaded}

\begin{verbatim}
## [1] TRUE
## [1] TRUE
\end{verbatim}

The p-value of both test are below the threshold, hence we
\textbf{reject} the null hypothesis that the mean are equal and conclude
that \textbf{dosage increases tooth growth}

\end{document}
